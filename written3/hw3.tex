\documentclass{article}
\begin{document}
\setcounter{tocdepth}{3}

\setcounter{section}{9}
\setcounter{subsection}{2}
\setcounter{subsubsection}{0}
\subsubsection{
a: there are 9 arcs\newline
b: 2 acyclic paths a-d, a-b-c-d\newline
c: a,e\newline
d: f,c\newline
e: a-b-f-a, a-b-c-d-e-f-a, b-c-d-e-b, a-d-e-f-a\newline
f: f-a-b-f-a-b-f (3 of this)(could be f-a-b-f), a-b-c-d-e-b-f-a (5 of this)(could be upper loop for a,f,b and lower loop for b,c,d,e), d-e-f-a-b-f-a-d (4 of this)(start with a,b,e,f)
}

\setcounter{section}{9}
\setcounter{subsection}{3}
\setcounter{subsubsection}{4}
\subsubsection{
Prove: $ \sum\limits_{i=0}^n def(V_i) == 2 \mid E_n \mid$ for a graph G = (V, E) and n $\geq{1}$ \newline
Base case:\newline
$V == \emptyset E == \emptyset $\newline
$deg( V) == 2 \vert D \vert$ \newline
$0 == 0$\newline
Assume: $ \sum\limits_{i=0}^k deg(V_i) == 2\vert E_k\vert$  for a graph G = (V, E) and n $\geq{1}$\newline
Prove: $ \sum\limits_{i=0}^{k+1} deg(V_i) == 2\vert E_{k+1} \vert$  for a graph G = (V, E) and n $\geq{1}$\newline
$ \sum\limits_{i=0}^{k} deg(V_i) + deg(V_{k+1}) $ by definition of sum \newline
$ 2 \vert E_k\vert + deg(V_{k+1}) $ by inductive hypothesis\newline
Here we must ask the question: What exactly does $deg(V_{k+1})$ equal? We know that for and undirected graph if $\exists E(u,v)$ then $\exists E(v,u)$. This is also true for a vertex pointing to itself (it gets counted twice). We also know that the edge set $E_k$ can't get smaller with the addition of vertex $V_{k+1}$. This means the addition of any edges by vertex $V_{k+1}$ may grow the set or may not grow the set. With all of this being said, $deg(V_{k+1})$ could equal to 0 or 2 or 4 ... or 2k verticies. In any case, this MUST be the new value of $ 2 \vert E_{k+1}\vert $\newline
$ 2 \vert E_k\vert + deg(V_{k+1})  == 2\vert E_{k+1} \vert$
}

\setcounter{section}{9}
\setcounter{subsection}{4}
\setcounter{subsubsection}{0}
\subsubsection{
Constructed Graph:\newline
Marquette\newline
I   I  Excabana\newline
I   Sault Ste. Marie\newline
Menominee\newline
.\newline
Grand Rapids \newline
I I I I I Battle Creek\newline
I I I I Detroit\newline
I I I Lansing\newline
I I Ann Arbor\newline
I Kalamazoo\newline
Saginov\newline
 I\newline
 Flint
}

\newpage

\setcounter{section}{1}
\setcounter{subsection}{2}
\subsection{
If my answer is TRUE, I must show that: for every shorest path tree, there exists an adjacency list that corresponds exactly to this tree\newline
If my answer is FALSE, I must show that: a counter example exists, namely there exists at least one representation of an adjacency list that does not yield a shortest path tree run from s.\newline
The answer is TRUE\newline
For a shortest path tree to be created, there must be edges connecting nodes. The edges are stored in the adjacency list. BFS utilizes these edges to do its job. No edges equals no BFS, which is the expected outcome. Because BFS must take a step by step approach, the ordering of the edges matters and a construction of a shortest path because of this order can yeild unique DBFS trees.
}

\newpage

\setcounter{section}{2}
\setcounter{subsection}{0}
\subsection{
A. TRUE, for a finite graph.\newline
suppose a graph with V of finite size n. first, assume that every node points to at least one other node (the (out)deg(ALL VERTICIES) != 0). take a vertex out of V. It can't point to itself, so it has n-1 choices to point to. In this example, there are n verticies and at least n edges. The edge set contains an edge $(u,v)$ where $u\neq v$. As the edges are enumerated, the last edge (one that doesn't have an outdegree), will have have an empty verticie set to point to and this resulting edge must create a cycle. If the last vertex did not point to anything (it has an (out)deg == 0, the graph remains a DAG and we are left with one vertex, u, where $deg(u), == 0 $
}

\subsection{
B. FALSE\newline
suppose the graph G with set V = {a,b,c,d} and E = {(a,b)(b,d)(d,a)}\newline
c has an (out)deg of zero and the graph is cyclic from a-b-d-a-d-b-...\newline
therefore, this statement is false.
}
\subsection{
C. TRUE\newline
see questions 1.C. the arguement shows that, in the addition of vertex, there will always be a last "vertex" that must point to another node which would cause a cycle. If all verticies belong to some outgoing edge, there are n edges but n-1 nodes to point to. 1 edge remains and this edge results in a cycle.
}

\end{document}
